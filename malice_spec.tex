\documentclass[a4wide, 11pt]{article}
\usepackage{a4, fullpage, amssymb}
\setlength{\parskip}{0.3cm}
\setlength{\parindent}{0cm}

% This is the preamble section where you can include extra packages etc.

\begin{document}

\title{MAlice Language Specification}

\author{Robert Kruszewski \and Piotr Bar \and Łukasz Koprowski}


\date{\today}         % inserts today's date

\maketitle            % generates the title from the data above

\section{Introduction}

This document constitutes the specification of MAlice programming language.
First the BNF of the language is given. The BNF is followed by explanation
of language constructs and their meaning, i.e. what different operators mean
and what is an expected behaviour of these. Further some internal workings of compiler
are explained as they would faciliate understanding particular decisions when
it comes to language design.

\section{BNF Grammar} 

\begin{verbatim}
      assignment  := id function_name type assign_term
                  | id function_name expression 
                  | 'Alice' function_call expression

      assign_term := separator
                  | separator assignment

      expression  := expression bin_op expression 
                  | un_op id
                  | id
                  | numeral

      type        := 'letter' | 'number'

      separator   := '.' | ',' | 'then' | 'or' | 'and' | 'too'

      bin_op      := '+'  | '-' | '/' | '%' | '^' | '&' | '|'
      un_op       := '~'
\end{verbatim}

\section{Semantics}

\subsection{Types}

From BNF you can easily see that MAlice supports two types: number and letter.

\begin{enumerate}
     \item
     number is a 64-bit floating point number. Due to this format restriction and lack of special way of representing big numbers this is also the limit to calculations results. However, nothing stops the programmer from implementing this feature by himself

     \item
     letter is a character and it occupies 8-bit of memory

\end{enumerate}

\subsection{Operators}

Operators have their usual meaning as used in other programming languages. All arithmetic operations are defined in the subset of $\mathbb{R}$ which is representable in 64-bit floating point representation.

\begin{itemize}

    \item
    \textit{+} - represents addition of two numbers, i.e. $ 2 + 3 = 5 $

    \item
    \textit{-} - represents subtraction of two numbers, i.e. $ 5 - 1 = 4 $. Given that and the type specification subtraction is allowed 

    \item
    \textit{/} - is division of two numbers which always returns a \emph{number}, i.e. $ 6 / 3 = 2.0 $. Bear in mind that compiler will treat $ 23 / 0 $ as valid expression. Such error will be a runtime exception

    \item
    \textit{\%} - means a modulo operator, i.e. $ 4 \% 3 = 1 $

    \item
    \textit{\^} - represents bitwise xor of two numbers, $ 0111011 ^ 0110010 = 0001001 $

    \item
    \textit{\&} - is used to represent bitwise and operation between two numbers,  $ 1101 \& 0110 = 0100 $

    \item
    \textit{|} - represents bitwise or of two numbers, i.e. $ 10110 | 01001 = 10111 $

    \item
    \textit{\~} - is a unary operator which gives the bitwise inverse, $ ~(010) = 101 $
    
\end{itemize}

When it comes to precedence of operators it is given by the following sequence

\begin{center} \~, /, \%, +, -, \&, \^, | \end{center}

Operators precedence decreases from left to right. Therefore \emph{\~} has highes precedence and \emph{|} lowest.
All of the operators are only valid for numbers as they meanings for letters would be unintuitive and best if would be left to the programmer to make the decision.

%Finally for this document, if you want to include a reference
%then you put it into a \texttt{thebibliography\{...\}}
%environment (see below in source file) and then 
%cite it like this \cite{lamport94}
%(you will need to run \texttt{latex} twice to get it to process the citation),
%or you can use BibTex but that is probably overkill for now.

\begin{thebibliography}{9}

\bibitem{lamport94}
  Leslie Lamport,
  \emph{\LaTeX: A Document Preparation System}.
  Addison Wesley, Massachusetts,
  2nd Edition,
  1994.

\end{thebibliography}


\end{document}
